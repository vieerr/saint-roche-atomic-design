%============================================%
% the following code are samples of different
% things used for the docs
%============================================%

%--------------- Common Utils ---------------%
“”                                         % Double Quotes for text rendered in PDF 
\textit {text}                             % Italic text
\textbf{text}                              % Bold text
\$                                         % Display dolar symbol
\href{url}{\color{blue}\underline{Text}}   % Hyperlink format
                                           % List of square dots
\begin{itemize}
    \item item1
    
    \item item2
    
    \item item3
\end{itemize}

%-------------- Adding an image --------------%
% define width and heigh with apropiate values
% with a label ---> Aparece con pie de página
\begin{figure}[H]
    \centering
    \includegraphics[width=0.6 \textwidth, height=8cm, keepaspectratio]{imagen1.png}
    \caption{Descripción de la ilustración}
    \label{fig:reference}
\end{figure}


% without a label ---> No aparece en el índice de figuras
\begin{figure}[H]
    \centering
    \includegraphics[width=0.6 \textwidth, height=8cm, keepaspectratio]{imagen1.png}
\end{figure}


%-------------- Cites in the text --------------%
% The reference must be in bibTex format in "ref.bib" file
\parencite{NameOfTheAuthor}


%-------------- Inline code bits --------------%
% Similar as Markdown: `code`
\lstinline{SmallCode}


%-------------- Dart code block --------------%
\begin{center}
\begin{lstlisting}
void main() {
    print('Hola Mundo');
    for (int i = 0; i < 5; i++) {
        print(i);
    }
}
\end{lstlisting}
\end{center}