\subsection{Atomic Design}
Es una metodología de diseño que parte de la idea simple de como se organizan los seres vivos. Su filosofía se basa en ir creando componentes reutilizables y personalizables para agilizar el desarrollo y favorecer la reutilización de componentes \parencite{Garca2024}. Todo esto siguiendo una estructura de diseño anteriormente preestablecida, como paletas de color, tamaño de texto, fuente y demás decisiones de diseño para aumentar la experiencia de usuario.

Sus fundamentos se basan en las siguientes estructuras:

\textbf{Átomos:} Son unidades básicas de diseño, tratándose de elementos sencillos como iconos, botones, inputs o etiquetas. Se llega a obtener un átomo cuando ser reconoce que no se lo puede descomponen en partes más pequeñas que esta.

\textbf{Moléculas:} Se obtienen al combinar distintos átomos, aunque siguen siendo de nivel elemental, puesto que puede ser reutilizable en varias partes de la aplicación. Un ejemplo podría ser composiciones definidas para una barra de búsqueda.

\textbf{Organismos:} Son componentes más grandes y complejos, puesto que incorporan varias moléculas, generalmente ya se tratan de piezas reconocibles dentro de la interfaz. Ejemplos pueden ser una barra de navegación o una tarjeta de un producto.

\textbf{Plantillas:} En este punto ya se define la estructura de una página o vista. Se encarga de organizar varios elementos a la vez para definir la disposición de los elementos en la pantalla.

\textbf{Páginas:} Esta es el nivel final, donde toma como base una plantilla y luego se la llena con contenido real para así presentarla al usuario.

\subsection{Navegación en Flutter}

