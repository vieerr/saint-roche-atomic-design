\subsection{Conclusiones}
\begin{itemize}
    \item La realización del laboratorio permitió aplicar el patrón de diseño MVC, y también la metodología Atomic Design, lo cual permitió tener un proyecto con una estructura organizada, clara y adaptable para las distintas soluciones que se proponen para los problemas del laboratorio.
    
    \item El desarrollo del proyecto utilizando widgets reutilizables y esquemas de color uniformes permitió desarrollar un proyecto con coherencia visual en todas las soluciones que se plantean; esto facilitó la creación de las plantillas consistentes y mejoró la experiencia de usuario.

    \item El uso de la navegación entre vistas para poder visitar las diferentes soluciones planteadas con el uso de widgets reutilizables demostró mayor flexibilidad para el desarrollo de este proyecto; además, se logró mantener una comunicación entre los diferentes componentes y el flujo lógico dentro de la app desarrollada.
\end{itemize}

\subsection{Recomendaciones}
\begin{itemize}
    \item Se recomienda aplicar patrones de diseño y metodologías como MVC y Atomic Design en otros proyectos para poder tener un desarrollo limpio, modular y fácilmente escalable, lo que permite ofrecer un producto de mayor calidad.

    \item Se recomienda hacer uso de widgets que puedan ser más personalizados y temáticos, ya que estos permiten generar componentes que son más reutilizables, además de que fortalecen la identidad visual y reducen la redundancia de código para tener un código más limpio y de mejor calidad.
    
    \item Se recomienda profundizar en las diferentes opciones de navegación que Flutter ofrece, además de las diferentes opciones de manejo de estados, para poder optimizar el rendimiento y mejorar la interacción entre las diferentes pantallas del proyecto.
\end{itemize}