% =====================
% --- Bloques de código ---
% =====================
% Definir colores para sintaxis Dart
\definecolor{keywordColor}{RGB}{0,0,180}   % azul para keywords
\definecolor{stringColor}{RGB}{163,21,21}  % rojo para strings
\definecolor{commentColor}{RGB}{0,128,0}   % verde para comentarios

% Configuración de listings para Dart
\lstdefinelanguage{Dart}{
  keywords={abstract, as, assert, async, await, break, case, catch, class, const,
            continue, covariant, default, defer, do, else, enum, export, extends,
            external, factory, false, final, finally, for, Function, get, if, implements,
            import, in, interface, is, late, library, mixin, new, null, on, operator,
            part, required, rethrow, return, set, show, static, super, switch, sync, 
            this, throw, true, try, typedef, var, void, while, with, yield},
  sensitive=true,
  comment=[l]{//},
  morecomment=[s]{/*}{*/},
  string=[b]",
}

\lstset{
  literate={á}{{\'a}}1
           {é}{{\'e}}1
           {í}{{\'i}}1
           {ó}{{\'o}}1
           {ú}{{\'u}}1
           {Á}{{\'A}}1
           {É}{{\'E}}1
           {Í}{{\'I}}1
           {Ó}{{\'O}}1
           {Ú}{{\'U}}1
           {ñ}{{\~n}}1
           {Ñ}{{\~N}}1
           {¡}{{!`}}1
           {¿}{{?`}}1,
  language=Dart,
  basicstyle=\ttfamily\fontsize{11}{13}\selectfont, % Consolas 11pt
  keywordstyle=\color{keywordColor}\bfseries,
  stringstyle=\color{stringColor},
  commentstyle=\color{commentColor}\itshape,
  showstringspaces=false,
  tabsize=4,
  breaklines=true,
  numbers=none,
  frame=single,
  rulecolor=\color{black},
  linewidth=\textwidth,
  framerule=0.1pt,
  xleftmargin=0pt,
  xrightmargin=0pt,
  captionpos=b
}

% Configuración de lstinline
\definecolor{inlinebg}{RGB}{240,240,240}       % gris claro
\definecolor{inlineborder}{RGB}{200,200,200}

\let\oldlstinline\lstinline
\renewcommand{\lstinline}[1]{%
  \colorbox{inlinebg!80!white}{%
    \setlength{\fboxsep}{1pt}%
    \textcolor{black}{\ttfamily\fontsize{11}{13}\selectfont #1}%
  }%
}

% =====================
% --- Bibliografía ---
% =====================
\addbibresource{ref.bib}  
\DeclareLanguageMapping{spanish}{spanish-apa}
\hypersetup{
    colorlinks=true,
    linkcolor=black,     % enlaces internos (capítulos, secciones) en negro
    citecolor=black,     % enlaces de citas en negro
    urlcolor=black       % enlaces de URLs en negro
}

% =====================
% --- Márgenes ---
% =====================
\geometry{
  top=2.54cm,
  bottom=2.54cm,
  left=3.17cm,
  right=3.17cm
}

% =====================
% --- Fuente Arial (Helvética) ---
% =====================
\usepackage{helvet}
\renewcommand{\familydefault}{\sfdefault}
\renewcommand\normalsize{\fontsize{11}{13}\selectfont}

% =====================
% --- Espaciado general ---
% =====================
\setstretch{1.5}
\setlength{\parindent}{0pt}
\setlength{\parskip}{16pt}

% =====================
% Encabezado y numeración
% =====================
\pagestyle{fancy}
\fancyhf{} % limpia encabezados/pies
\fancyhead[R]{\thepage} % número de página arriba a la derecha
\renewcommand{\headrulewidth}{0pt} % sin línea inferior

% =====================
% Colores institucionales
% =====================
\definecolor{azulOscuro}{RGB}{54,95,145}  % para Título1
\definecolor{azulClaro}{RGB}{79,129,189}  % para Título2
\definecolor{negro}{RGB}{0,0,0}

% =====================
% --- Formato de Figuras ---
% =====================
\captionsetup[figure]{
    labelfont={bf, color=azulClaro},
    textfont={color=azulClaro, small},
    justification=centering,
    labelsep=space
}

\graphicspath{{img/}}

% =====================
% Formato de Título1 (chapters) y Título2 (sections)
% =====================

% -- Título 1 --
\titleformat{\section}[hang]
  {\bfseries\color{azulOscuro}\fontsize{14}{16}\selectfont}
  {\thesection.}
  {1em}
  {}
\titlespacing*{\section}{0pt}{24pt}{{-16pt}}  % espaciado "antes" 24pt, "después" 12pt

% -- Título 2 --
\titleformat{\subsection}[hang]
  {\bfseries\color{azulClaro}\fontsize{12}{14}\selectfont}
  {\thesubsection.}
  {0.8em}
  {}
\titlespacing*{\subsection}{0pt}{10pt}{{-16pt}}  % espaciado "antes" 10pt, "después" 0pt

% =====================
% Índice de contenidos dinámico
% =====================
\setcounter{tocdepth}{2} % muestra capítulos y secciones en el índice
\hypersetup{
    colorlinks=true,
    linkcolor=negro,
    urlcolor=negro
}

% Hacer que las secciones tengan puntos hasta el número de página
\renewcommand{\cftsecleader}{\cftdotfill{\cftdotsep}}


% =====================
% Índice de figuras dinámico
% =====================
\renewcommand{\cftfigpresnum}{Ilustración\ }
\setlength{\cftfignumwidth}{3cm}
\renewcommand{\cftfigleader}{\cftdotfill{\cftdotsep}}