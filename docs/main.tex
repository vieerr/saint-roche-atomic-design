\documentclass[12pt,letterpaper]{article}

% =====================
% Configuración general
% =====================
\usepackage[utf8]{inputenc}
\usepackage[T1]{fontenc}
\usepackage[spanish]{babel}
\usepackage{listings}
\usepackage{xcolor}
\usepackage{geometry}
\usepackage{setspace}
\usepackage{titlesec}
\usepackage{fancyhdr}
\usepackage{hyperref}
\usepackage{titlesec}
\usepackage{tabularx}
\usepackage{tocloft}
\usepackage{graphicx}
\usepackage{caption}
\usepackage{tocloft}
\usepackage{float}
\usepackage[backend=biber,style=apa,natbib=true]{biblatex}
% =====================
% --- Bloques de código ---
% =====================
% Definir colores para sintaxis Dart
\definecolor{keywordColor}{RGB}{0,0,180}   % azul para keywords
\definecolor{stringColor}{RGB}{163,21,21}  % rojo para strings
\definecolor{commentColor}{RGB}{0,128,0}   % verde para comentarios

% Configuración de listings para Dart
\lstdefinelanguage{Dart}{
  keywords={abstract, as, assert, async, await, break, case, catch, class, const,
            continue, covariant, default, defer, do, else, enum, export, extends,
            external, factory, false, final, finally, for, Function, get, if, implements,
            import, in, interface, is, late, library, mixin, new, null, on, operator,
            part, required, rethrow, return, set, show, static, super, switch, sync, 
            this, throw, true, try, typedef, var, void, while, with, yield},
  sensitive=true,
  comment=[l]{//},
  morecomment=[s]{/*}{*/},
  string=[b]",
}

\lstset{
  literate={á}{{\'a}}1
           {é}{{\'e}}1
           {í}{{\'i}}1
           {ó}{{\'o}}1
           {ú}{{\'u}}1
           {Á}{{\'A}}1
           {É}{{\'E}}1
           {Í}{{\'I}}1
           {Ó}{{\'O}}1
           {Ú}{{\'U}}1
           {ñ}{{\~n}}1
           {Ñ}{{\~N}}1
           {¡}{{!`}}1
           {¿}{{?`}}1,
  language=Dart,
  basicstyle=\ttfamily\fontsize{11}{13}\selectfont, % Consolas 11pt
  keywordstyle=\color{keywordColor}\bfseries,
  stringstyle=\color{stringColor},
  commentstyle=\color{commentColor}\itshape,
  showstringspaces=false,
  tabsize=4,
  breaklines=true,
  numbers=none,
  frame=single,
  rulecolor=\color{black},
  linewidth=\textwidth,
  framerule=0.1pt,
  xleftmargin=0pt,
  xrightmargin=0pt,
  captionpos=b
}

% Configuración de lstinline
\definecolor{inlinebg}{RGB}{240,240,240}       % gris claro
\definecolor{inlineborder}{RGB}{200,200,200}

\let\oldlstinline\lstinline
\renewcommand{\lstinline}[1]{%
  \colorbox{inlinebg!80!white}{%
    \setlength{\fboxsep}{1pt}%
    \textcolor{black}{\ttfamily\fontsize{11}{13}\selectfont #1}%
  }%
}

% =====================
% --- Bibliografía ---
% =====================
\addbibresource{ref.bib}  
\DeclareLanguageMapping{spanish}{spanish-apa}
\hypersetup{
    colorlinks=true,
    linkcolor=black,     % enlaces internos (capítulos, secciones) en negro
    citecolor=black,     % enlaces de citas en negro
    urlcolor=black       % enlaces de URLs en negro
}

% =====================
% --- Márgenes ---
% =====================
\geometry{
  top=2.54cm,
  bottom=2.54cm,
  left=3.17cm,
  right=3.17cm
}

% =====================
% --- Fuente Arial (Helvética) ---
% =====================
\usepackage{helvet}
\renewcommand{\familydefault}{\sfdefault}
\renewcommand\normalsize{\fontsize{11}{13}\selectfont}

% =====================
% --- Espaciado general ---
% =====================
\setstretch{1.5}
\setlength{\parindent}{0pt}
\setlength{\parskip}{16pt}

% =====================
% Encabezado y numeración
% =====================
\pagestyle{fancy}
\fancyhf{} % limpia encabezados/pies
\fancyhead[R]{\thepage} % número de página arriba a la derecha
\renewcommand{\headrulewidth}{0pt} % sin línea inferior

% =====================
% Colores institucionales
% =====================
\definecolor{azulOscuro}{RGB}{54,95,145}  % para Título1
\definecolor{azulClaro}{RGB}{79,129,189}  % para Título2
\definecolor{negro}{RGB}{0,0,0}

% =====================
% --- Formato de Figuras ---
% =====================
\captionsetup[figure]{
    labelfont={bf, color=azulClaro},
    textfont={color=azulClaro, small},
    justification=centering,
    labelsep=space
}

\graphicspath{{img/}}

% =====================
% Formato de Título1 (chapters) y Título2 (sections)
% =====================

% -- Título 1 --
\titleformat{\section}[hang]
  {\bfseries\color{azulOscuro}\fontsize{14}{16}\selectfont}
  {\thesection.}
  {1em}
  {}
\titlespacing*{\section}{0pt}{24pt}{{-16pt}}  % espaciado "antes" 24pt, "después" 12pt

% -- Título 2 --
\titleformat{\subsection}[hang]
  {\bfseries\color{azulClaro}\fontsize{12}{14}\selectfont}
  {\thesubsection.}
  {0.8em}
  {}
\titlespacing*{\subsection}{0pt}{10pt}{{-16pt}}  % espaciado "antes" 10pt, "después" 0pt

% =====================
% Índice de contenidos dinámico
% =====================
\setcounter{tocdepth}{2} % muestra capítulos y secciones en el índice
\hypersetup{
    colorlinks=true,
    linkcolor=negro,
    urlcolor=negro
}

% Hacer que las secciones tengan puntos hasta el número de página
\renewcommand{\cftsecleader}{\cftdotfill{\cftdotsep}}


% =====================
% Índice de figuras dinámico
% =====================
\renewcommand{\cftfigpresnum}{Ilustración\ }
\setlength{\cftfignumwidth}{3cm}
\renewcommand{\cftfigleader}{\cftdotfill{\cftdotsep}}

% =====================
% Documento
% =====================
\begin{document}
\renewcommand{\figurename}{Ilustración} 
{\textbf{\textcolor{azulOscuro}{INFORME LABORATORIO 2}}}

%===========================================================
%===========================================================
% --- Tabla de Datos ---
%===========================================================
%===========================================================

{\setlength{\parskip}{0pt}
\section{Datos Generales}
}

\begin{tabularx}{\textwidth}{|>{\raggedright\arraybackslash}X|>{\raggedright\arraybackslash}X|}
\hline
Título del Informe: & Creación y ejecución de un proyecto \\
\hline
Autor(a): & Carlos Hernández, Olivier Paspuel, Antonio Revilla, Frederick Tipán\\
\hline
Carrera: & Ingeniería en Software \\
\hline
Asignatura o Proyecto: & Aplicar Paradigma Atomic Desing y componentes UI \\
\hline
Tutor o Supervisor: & Mgtr. Doris Karina Chicaiza Angamarca\\
\hline
Institución: & Universidad de las Fuerzas Armadas ESPE – Matriz Sangolquí \\
\hline
Fecha de entrega: & 31 de octubre de 2025 \\
\hline
\end{tabularx}


%===========================================================
%===========================================================
% --- Introducción ---
%===========================================================
%===========================================================

\section{Introducción}

Párrafo 1: Introducción al tema

Párrafo 2:  hablar de herramientas

Párrafo 3: especie de conclusión 

\newpage

%===========================================================
%===========================================================
% --- índice de Contenidos y Figuras ---
%===========================================================
%===========================================================

\renewcommand{\contentsname}{Índice de Contenidos}
{\setlength{\parskip}{0pt}
\tableofcontents
}

\vspace{1.0cm}

\renewcommand{\listfigurename}{Índice de Ilustraciones}
{\setlength{\cftbeforefigskip}{2pt}
\listoffigures
}

\newpage

%===========================================================
%===========================================================
% --- Objetivos ---
%===========================================================
%===========================================================

\section{Objetivos}
\subsection{Objetivo General}
OBJECT

\subsection{Objetivo Específicos}
- OBJECT

- OBJECT

- OBJECT

%===========================================================
%===========================================================
% --- Marco Teórico ---
%===========================================================
%===========================================================

\section{Marco Teórico}
\subsection{Paradigma Atomic Design}
MARCO TEORICO

%===========================================================
%===========================================================
% --- Desarollo ---
%===========================================================
%===========================================================

\section{Desarrollo}

El proyecto se encuentra cargado en el siguiente repositorio de GitHub: 

\href{https://github.com/vieerr/saint-roche-atomic-design}{\color{blue}\underline{https://github.com/vieerr/saint-roche-atomic-design}}

\subsection{Problema 1}

\textit {“Desarrollo y explicación de problema 1 “Indicaciones al ejercicio en clase calcular el iva de 15\%, si las compras superan \textdollar2000 al cliente se agregar un descuento del 20\%, mostrar tanto el sueldo a ganar del vendedor y la factura de la venta de productos.”}

\textbf{Modelo}

Para desarrollar este ejercicio, se tienen 3 métodos en la clase del modelo.

\begin{center}
\begin{lstlisting}
  // Getter total ventas
  double get totalVentas => venta1 + venta2 + venta3;

  double calcularSueldo() {
    double sueldoBase = 36500;
    double comision = totalVentas * 0.12;
    return sueldoBase + comision;
  }

  double obtenerDescuento() {
    if(totalVentas >= 2000){
      return totalVentas * 0.20;
    }
    return 0;
  }
\end{lstlisting}
\end{center}

Las variables \lstinline{venta1}, \lstinline{venta2} y \lstinline{venta3} son atributos de la clase modal. Estas permiten obtener los valores del total de las ventas, un descuento si las ventas totales supera los \textdollar2000 y el sueldo del trabajador más su comisión del 12\% por cada venta.

\textbf{Controlador}

El controlador, toma este modelo y realiza las validaciones de texto correspondientes antes de ejecutar acciones sobre las variables numéricas de las ventas. En este método se retorna los valores respecto al total de las ventas, sueldo, descuento y el mensaje de error (en caso de haberlo).

\begin{center}
\begin{lstlisting}
  final billModel = BillModel(v1, v2, v3);
   final sueldo = billModel.calcularSueldo();
   final descuento = billModel.obtenerDescuento();

   return (billModel.totalVentas, sueldo, descuento, "");
\end{lstlisting}
\end{center}

\textbf{Vistas}

Dentro de la vista, una vez que se presiona el botón de calcular, se utiliza el controlador para mostrar obtener los resultados o los errores antes de realizar la acción. En caso de que todo fluya con normalidad, se pasan los datos a la ruta de resultados para mostrar los resultados en una nueva pantalla.

\begin{center}
\begin{lstlisting}
void _calcular() {
  final (totalVentas, sueldo, descuento, err) = mainctrl.calcularDatos(
      venta1ctrl.text,
      venta2ctrl.text,
      venta3ctrl.text,
  );

  // actualizar mensaje de error
  setState(() {
    msgErr = err;
  });

  // no mostrar pasar a la siguiente pantalla si hubo un error
  if(err != "") {
    return;
  }

  Navigator.pushNamed(
      context, "/bill/result",
      arguments: {
        'totalVentas': totalVentas,
        'sueldo': sueldo,
        'descuento': descuento,
      }
  );
}
\end{lstlisting}
\end{center}

Puesto que se mandaron argumentos en forma de clave valor, se realizaron colocaron los siguientes comandos para obtener cada uno de los argumentos en la vista de resutlados.

\begin{center}
\begin{lstlisting}
  @override
  Widget build(BuildContext context) {
    final args = ModalRoute.of(context)!.settings.arguments as Map<String, double>;

    final double totalVentas = args['totalVentas']!;
    final double sueldo = args['sueldo']!;
    final double descuento = args['descuento']!;
  }
\end{lstlisting}
\end{center}

\textbf{Ejecución}

De esta manera se pudo cumplir con el ejercicio propuesto, donde se toman 3 valores y con ellos se realizan varios cálculos mostrados en la siguiente pantalla:

\begin{figure}[H]
    \centering
    \includegraphics[width=0.8 \textwidth, height=10cm, keepaspectratio]{ejecucion_ej1.png}
    \caption{Ejecucción ejercicio 1}
    \label{fig:ej1_ejecuccion}
\end{figure}
\subsection{Problema 2}

\textit {“Un profesor tiene un salario inicial de \$1500, y percibe un incremento de 10\% anual durante 6 años. ¿Cuál es su salario al cabo de 6 años? ¿Qué salario ha recibido en cada uno de los 6 años?”}


\subsection{Problema 3}

\textit {“El náufrago satisfecho ofrece hamburguesas sencillas (S), dobles (D) y triples (T), las cuales tienen un costo de \$20, \$25 y \$28 respectivamente. La empresa acepta tarjetas de crédito con un cargo de 5\% sobre la compra. Suponiendo que los clientes adquieren N hamburguesas, las cuales pueden ser de diferente tipo, realice un algoritmo para determinar cuánto deben pagar.”}

\textbf{Modelo}

Puesto que se tienen 3 opciones predefinidas, se optó por realizar un \lstinline{enum} con las 3 opciones especificadas.

\begin{center}
\begin{lstlisting}
enum HamType {
  sencilla('Sencilla', 20),
  doble('Doble', 25),
  triple('Triple', 28);

  final String name;
  final double price;
  const HamType(this.name, this.price);
}
\end{lstlisting}
\end{center}

Con este enum, se lo utilizó dentro del modelo, donde se especifica una cantidad para cada tipo de hamburguesa. Adicionalmente se tienen los métodos para calular el total tomando en cuenta el precio base y la cantidad correspondiente.

\begin{center}
\begin{lstlisting}
class HamburgerModel {
  final HamType hamburger;
  int quantity;

  HamburgerModel({required this.hamburger, this.quantity = 0});

  double get total => hamburger.price * quantity;
}
\end{lstlisting}
\end{center}

\textbf{Controlador}

Primero se tiene una lista con el modelo, por lo que se lo transforma mediante:

\begin{center}
\begin{lstlisting}
final List<HamburgerModel> hamburgers = [
  for(var ham in HamType.values)
    HamburgerModel(hamburger: ham)
];
\end{lstlisting}
\end{center}

Luego se definieron métodos para aumentar y disminuir la cantidad de un determinado tipo de hamburguesa. Estos métodos serán utilizados por los widgets \lstinline{Counter} para mostrar en pantalla la cantidad actual de cada hamburguesa.

\begin{center}
\begin{lstlisting}
  void increaseQty(HamburgerModel ham) {
    ham.quantity++;
  }

  void decreaseQty(HamburgerModel ham) {
    if (ham.quantity > 0) {
      ham.quantity--;
    }
  }
\end{lstlisting}
\end{center}

Luego se tienen los métodos para calcular el total a pagar, esto se lo hace recorriendo la lista principal y llamando al método \lstinline{total} para sumar acumulativamente los valores.

\begin{center}
\begin{lstlisting}
double get total {
  double hamTotal = 0;
  for(int i=0; i<hamburgers.length; i++) {
    hamTotal += hamburgers[i].total;
  }
  return hamTotal;
}
\end{lstlisting}
\end{center}

Luego, en caso de que se tenga recargo por la tarjeta de crédito, se obtiene el valor del recargo en base al total de toda la venta. Adicionalmente se tiene con un método que verifica si no se ha seleccionada ninguna hamburguesa, si este es el caso, se devolverá un mensaje de error.

\begin{center}
\begin{lstlisting}
double get charge {
    return total * RECARGO;
  }

String allHamburgersZero() {
  for(int i=0; i<hamburgers.length; i++){
    if(hamburgers[i].quantity != 0) {
      return "";
    }
  }
  return "Agregue por lo menos una hamburguesa";
}
\end{lstlisting}
\end{center}

\textbf{Vistas}

Dentro de la vista, se utiliza el controlador, junto con variables para el manejo del mensaje de error y si se ha aplicado el método de pago por tarjeta de crédito.

\begin{center}
\begin{lstlisting}
final hamCtrl = HamburgerController();
bool _isCreditCardPayment = false;
String _errMsg = "";
\end{lstlisting}
\end{center}

Cuando el usuario presiona el botón para calcular el pago se utiliza el controlador, luego se verifica si se ha seleccionado al menos una hamburguesa, luego se comprueba si hay un recargo si se pagó con tarjeta de crédito, para finalmente pasar a la pantalla de resutlados con los argumentos correspondientes.

\begin{center}
\begin{lstlisting}
void _computeAllHamburgers() {
 setState(() {
   _errMsg = hamCtrl.allHamburgersZero();
 });

  if(_errMsg != "") {
    return;
  }

  double charge = 0;
  if(_isCreditCardPayment){
    charge = hamCtrl.charge;
  }

  Navigator.pushNamed(
    context, "/hamburger/result",
    arguments: {
      'total': hamCtrl.total,
      'recargo': charge,
      'hamburgesas': hamCtrl.hamburgers,
    }
  );
}
\end{lstlisting}
\end{center}

Luego, en la pantalla de resultados, se recuperaron los argumentos mediante:

\begin{center}
\begin{lstlisting}
final args = ModalRoute.of(context)!.settings.arguments as Map<String, dynamic>;

final double total = args['total']!;
final double recargo = args['recargo']!;
final List<HamburgerModel> hamburguesas = args['hamburgesas']!;
\end{lstlisting}
\end{center}

\textbf{Ejecución}

De esta manera se pudo cumplir con el ejercicio propuesto, donde se dan a mostrar 3 opciones de hamburguesa con la posibilidad de poder realizar el pago con tarjeta con una comisión adicional mediante las siguientes pantallas:

\begin{figure}[H]
    \centering
    \includegraphics[width=0.8 \textwidth, height=10cm, keepaspectratio]{ejecucion_ej3.png}
    \caption{Ejecucción ejercicio 3}
    \label{fig:ej3_ejecuccion}
\end{figure}
\subsection{Problema 4}

\textit {“Se requiere un algoritmo para determinar, de N cantidades, cuántas son de cero, cuántas son menores a cero, y cuántas son mayores a cero.”}

\textbf{Modelo}

El modelo recibe una lista de números y cuenta cuántos son ceros, negativos y positivos. Para ello se definen tres getters que recorren la lista.

\begin{center}
\begin{lstlisting}
int get zeros {
  int count = 0;
  for (int i = 0; i < numbers.length; i++) {
    if (numbers[i] == 0) {
      count++;
    }
  }
  return count;
}

int get negatives {
  int count = 0;
  for (int i = 0; i < numbers.length; i++) {
    if (numbers[i] < 0) {
      count++;
    }
  }
  return count;
}

int get positives {
  int count = 0;
  for (int i = 0; i < numbers.length; i++) {
    if (numbers[i] > 0) {
      count++;
    }
  }
  return count;
}
\end{lstlisting}
\end{center}

Cada getter itera sobre la lista de números y verifica la condición correspondiente, incrementando un contador cuando se cumple.

\textbf{Controlador}

El controlador valida los datos ingresados antes de procesarlos. Primero verifica que se haya ingresado la cantidad de números a procesar.

\begin{center}
\begin{lstlisting}
if (cantidad.isEmpty) {
  String err = "Ingrese la cantidad de números";
  return (0, 0, 0, err);
}

final n = int.tryParse(cantidad);

if (n == null) {
  String err = "Ingrese un valor numérico válido";
  return (0, 0, 0, err);
}

if (n <= 0) {
  String err = "La cantidad debe ser mayor a cero";
  return (0, 0, 0, err);
}
\end{lstlisting}
\end{center}

Luego valida que se hayan ingresado exactamente la cantidad de valores indicada y que todos sean numéricos.

\begin{center}
\begin{lstlisting}
if (valores.length != n) {
  String err = "Debe ingresar exactamente $n números";
  return (0, 0, 0, err);
}

List<double> numeros = [];
for (int i = 0; i < valores.length; i++) {
  if (valores[i].isEmpty) {
    String err = "Complete todos los campos";
    return (0, 0, 0, err);
  }

  final valor = double.tryParse(valores[i]);
  if (valor == null) {
    String err = "Ingrese valores numéricos válidos";
    return (0, 0, 0, err);
  }

  numeros.add(valor);
}
\end{lstlisting}
\end{center}

Finalmente, crea una instancia del modelo y retorna los resultados.

\begin{center}
\begin{lstlisting}
final model = NumbersModel(numeros);
return (model.zeros, model.negatives, model.positives, "");
\end{lstlisting}
\end{center}

\textbf{Vistas}

La vista principal permite ingresar la cantidad de números a procesar. Al presionar el botón de generar campos, se crean dinámicamente los inputs necesarios.

\begin{center}
\begin{lstlisting}
void _actualizarCantidad() {
  final n = int.tryParse(cantidadCtrl.text);
  if (n != null && n > 0 && n <= 20) {
    setState(() {
      cantidad = n;
      valoresCtrl = List.generate(n, (index) => TextEditingController());
      msgErr = "";
    });
  } else if (n != null && n > 20) {
    setState(() {
      msgErr = "La cantidad no puede ser mayor a 20";
      cantidad = 0;
      valoresCtrl = [];
    });
  }
}
\end{lstlisting}
\end{center}

Una vez ingresados todos los valores, se procesan mediante el controlador.

\begin{center}
\begin{lstlisting}
void _calcular() {
  List<String> valores = valoresCtrl.map((c) => c.text).toList();
  final (zeros, negatives, positives, err) = ctrl.procesarNumeros(
    cantidadCtrl.text,
    valores,
  );

  setState(() {
    msgErr = err;
  });

  if (err != "") {
    return;
  }

  Navigator.pushNamed(
    context,
    "/numbers/result",
    arguments: {
      'zeros': zeros,
      'negatives': negatives,
      'positives': positives,
    },
  );
}
\end{lstlisting}
\end{center}

En la pantalla de resultados, se obtienen los argumentos pasados y se muestran en una tarjeta con el conteo de cada categoría.

\begin{center}
\begin{lstlisting}
@override
Widget build(BuildContext context) {
  final args = ModalRoute.of(context)!.settings.arguments as Map<String, int>;

  final int zeros = args['zeros']!;
  final int negatives = args['negatives']!;
  final int positives = args['positives']!;
  final int total = zeros + negatives + positives;
}
\end{lstlisting}
\end{center}

\textbf{Ejecución}

El ejercicio permite ingresar N cantidades y clasifica cada una según su valor. Los resultados muestran cuántos números son cero, menores a cero y mayores a cero.

\begin{figure}[H]
    \centering
    \includegraphics[width=0.8 \textwidth, height=9cm, keepaspectratio]{ejecucion_ej4.png}
    \caption{Ejecución ejercicio 4}
    \label{fig:ej4_ejecuccion}
\end{figure}
\subsection{Problema 5}

\textit {“Realice un algoritmo para determinar cuánto pagará una persona que adquiere N artículos, los cuales están de promoción. Considere que si su precio es mayor o igual a \$200 se le aplica un descuento de 15\%, y si su precio es mayor a \$100 pero menor a \$200, el descuento es de 12\%; de lo contrario, sólo se le aplica 10\%. Se debe saber cuál es el costo y el descuento que tendrá cada uno de los artículos y finalmente cuánto se pagará por todos los artículos obtenidos.”}

\textbf{Paso 1: Definición de arquitectura y metodología}

Primero para el desarrollo del problema se creo una arquitectura MVC considerando que se va a utilizar una metodología de atomic design para esto se crearon 4 carpetas las cuales son:
\begin{itemize}
    \item Controller
    \item Model
    \item View
    \item Widgets
\end{itemize}

\textbf{Paso 2: Creación de archivos y definición de modelos}

Con esta estructura creada se crearon diferentes archivos .dart para la solución del problema dentro de los archivos se crearon: 
\begin{itemize}
    \item articulo\_controller.dart
    \item articulo\_model.dart
    \item articulo\_view.dart
    \item articulo\_card.dart
\end{itemize}

\textbf{articulo\_model.dart}

En el primero archivo que es articulo\_model.dart se definió la estructura de datos de un artículo o productos dentro de la app que se desarrollo para resolver el problema. Dentro del archivo se creó la clase ArticuloModel que tiene algunos atributos como el precio, el descuento que va a recibir y el total, además dentro de esta clase se crea una funcio llamada calcular esta es la que permite construir un articulo completo con el precio orginal, el código de forma genera lo que hace es definir el porcentaje de descuento que se va aplicar atraves de sentencias Ifs anidades, también hace los cálculos del monto de descuento en dólares que va a tener un articulo y también hace la respta respectiva del precio  - descuento para poder obtener el valor final aplicado el descuento, al final esta clase empaqueta los 3 atributos en un objeto llamado articulo.


\begin{center}
\begin{lstlisting}
class Articulo {
  final double precio;
  final double descuento;
  final double total;

  Articulo({
    required this.precio,
    required this.descuento,
    required this.total,
  });
\end{lstlisting}
\end{center}

\textbf{articulo\_controller.dart}

Después en el archivo, se utilizo el respectivo modelo ya creado en esta se crea una clase llamada ArticuloController en la cual se crea una lista privada a la cual se le llamo \_artículos aquí se guardan los artículos que se van agregando dentro de la aplicación, esta case tiene algunas funciones principales entre estas las mas importantes son agregar artículos la cual trabaja con el modelo internamiente esta llama a Articulo.calcular con esta función se devuelve un objeto completa con su precio y descuento ya calculado dentro del modelo, y el controlador añade el arituclo a lista \_artículos, otra función importante es calcularTotal la cual revisa y toma todos los artículos que se guardan en la lista, de cada uno de los arituculos almacenados en la lista se va sumando el total que es el valor con el descuento aplicado y suma estos valores para devolver el total de la compra o de todos los artículos que se registran en la lista, otra función importante es limpiar la cual vacia la lista borrando todos los artículos que se agregaron en la lista.


\begin{center}
\begin{lstlisting}
  final List<Articulo> _articulos = [];
  
  List<Articulo> get articulos => _articulos;
  
  void agregarArticulo(double precio) {
    _articulos.add(Articulo.calcular(precio));
  }
  void limpiar() {
    _articulos.clear();
  }
  double calcularTotal() {
    return _articulos.fold(0, (sum, item) => sum + item.total);
  }
\end{lstlisting}
\end{center}

\textbf{articuloCard}

El archivo se creo con el propósito de poder dibujar una Tarjeta que permita mostrar un articulo considerando que se va a guardar como un objeto, la idea de esta clases es que poder mostrar los artículos para esto utiliza el modelo es por medio del modelo por el que recibe un objeto Articulo y de este se saca el precio, descuento y el total, para la tarjeta se usa la librería que se importa, por lo que no se utilizan colores estándar de flutter, si no que se definieron colores definidos dentro de nuestro proyecto para trabajar con un paleta de colores definida en todos los problemas, algo adicional que se hace en esta clase es que se calcular el porcentaje de descuento para mostrar en la tarjeta.

\textbf{articulo\_view.dart}

Por ultimo en el archivo, aquí es donde se unen los códigos anteriores, para poder crear la interfaz final que se va a mostrar al usuario final dentro de este archivo se crea un StatefulWidget, la cual permite hacer una actualización cada ves que se ingresa un nuevo producto, de forma general tiene cuatro partes importantes, la Entrada de datos el cual muestra un campo de texto con un UltraInput para poder ingresar el precio y también muestra el botón para agregar y limpiar artículos, después se tiene la lista de artículos que es la parte principal esta revisa los artículos del controller y si no se tiene artículos se muestra el mensaje "No hay artículos agregados", pero si la lista tiene artículos se usa un ListView.builder para crear un ArticuloCard por cada uno de los artículos que se tenga en la lista del controlador, otra parte importante de la View es el Resumen total el cual llama de forma constante a controlle.calcularTotal que muestra el total a pagar por cada articulo agregado, este valor se actualiza, y por ultimo se tiene una gestión de estados y de lógica de forma resumida se crea una instancia de ArticuloController la cual se usa para guardar el estado de la aplicación, también se tiene la lógica del botón agregar en la cual se valida que el numero sea valido caso contrario mostrara un error y si es valido se llama a controller.agregarArticulo, también se tiene la lógica del botón de limpiar para poder eliminar los artículos de la lista, algo importante dentro de la lógica de los botones es el setState la cual cada ves que se presiona se vuelve a graficar la pantalla o se actualiza para poder mostrar los artículos actualizados tanto si se limpiaron o si se agregaron.


\textbf{Paso 3: Integración en la aplicación}

Una ves que se programo una solución utilizando la metodologia atomic design y el patron MVC para el problema se procedio a instanciar en el main.dart y en el home.dart que permiten la navegación entre los programas y también mostrar el la plantilla que se creo. 

Lo primero que se realizo, en main.dart es importar el package que corresponde a products, específicamente la vista articulo\_view.dart, para de esta forma poder instanciarla en el MaterialApp. Después, se registra la ruta por su nombre '/articulos', esto se realiza  dentro del routes del MaterialApp, y de esta forma se asocia al widget ArticuloView, de tal forma que permite la navegación mediante rutas nombradas.

Despues en el archivo home.dart, lo que se realizo es la modificación el quinto HomeButton para cambiar el ícono a Icons.calculate\_rounded que de alguna manera interpreta una calculardora que tiene algo de relación con el problema. Además también, en el método onClick, se utilizo Navigator.pushNamed para definir la navegacion con nombres de ruta para este se define '/articulos', de esta forma al presionar el botón, el usuario va a navegar hacia la pantalla ArticuloView.

De esta manera, se logra integrar la nueva funcionalidad de cálculo de promociones en la aplicación, asegurando que el Home actúe como menú central y la navegación sea clara, directa y reutilizable.


\begin{center}
\begin{lstlisting}
// MAIN
'/products': (context) => const ArticuloView(),

// HOME  
icon: Icons.calculate_rounded,
title: "Promociones",
onClick:  () => Navigator.pushNamed(context, '/products'),
\end{lstlisting}
\end{center}

\textbf{Paso 4: Pruebas y verificación}

Por ultimo para verificar el funcionamiento se ejecuto en el navegador y se realizaron algunas pruebas ingresando diferentes valores 250 - 150 - 100 - 50 y al final todos los resultados fueron correctos.



%===========================================================
%===========================================================
% --- Conclusiones y Recomendaciones ---
%===========================================================
%===========================================================

\section{Conclusiones y Recomendaciones}
\subsection{Conclusiones}
OBJECT

\subsection{Recomendaciones}
OBJECT

%===========================================================
%===========================================================
% --- Referencias Bibliográficas ---
%===========================================================
%===========================================================

\section{Referencias Bibliográficas}
\printbibliography[heading=none]

%===========================================================
%===========================================================
% --- Anexos ---
%===========================================================
%===========================================================

\section{Anexos}

\end{document}
